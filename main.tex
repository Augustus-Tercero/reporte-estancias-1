\documentclass[12pt]{article}
\usepackage[left=2.5cm,top=2.0cm,right=2.5cm,bottom=3.0cm]{geometry}
\usepackage[utf8]{inputenc}
\usepackage[spanish]{babel}
\usepackage{amssymb, amsmath, amsbsy} % simbolitos
\usepackage{longtable} % para tablas largas
\usepackage{graphicx}
\usepackage{fancyhdr}
\usepackage{xcolor}
\usepackage{multirow}
\usepackage{listings}
\usepackage{caption}
\usepackage{subcaption}
\usepackage{pdfpages} % Incluir PDF en documento en LATEX
\usepackage{verbatim} % comentarios
\usepackage{algpseudocode}
\usepackage{algorithm}
\usepackage{multirow}
\usepackage{afterpage}
\usepackage{array,booktabs,ragged2e}
%\newcolumntype{R}[1]{>{\RaggedLeft\arraybackslash}p{#1}}
\newcolumntype{L}[1]{>{\raggedright\let\newline\\\arraybackslash\hspace{0pt}}m{#1}}
\newcolumntype{C}[1]{>{\centering\let\newline\\\arraybackslash\hspace{0pt}}m{#1}}
\newcolumntype{R}[1]{>{\raggedleft\let\newline\\\arraybackslash\hspace{0pt}}m{#1}}

\floatname{algorithm}{Algoritmo}
\renewcommand{\listalgorithmname}{Lista de algoritmos}
\renewcommand{\algorithmicrequire}{\textbf{Entrada:}}
\renewcommand{\algorithmicensure}{\textbf{Salida:}}


\usepackage[backend=bibtex,sorting=none]{biblatex}
% Estas lineas permiten romper los hipervinculos muy largos en las referencias!!!!
\setcounter{biburllcpenalty}{7000}
\setcounter{biburlucpenalty}{8000}
\addbibresource{referencias.bib} % ARCHIVO DE BIBLIOGRAFÍA


%\usepackage{url}
\usepackage[bookmarks=true,breaklinks=true,bookmarksopen=false,colorlinks=true,linkcolor=blue]{hyperref}
\usepackage[hyphenbreaks]{breakurl}
% Regla que define explicitamente que caracteres rompen los hipervinculos para separar las lineas
%https://es.overleaf.com/11089898rhgykrqyqytx
% Actualiza en automático la fecha de las citas de internet a la fecha de la compilación del documento
\usepackage{datetime}
\newdateformat{specialdate}{\twodigit{\THEDAY}-\twodigit{\THEMONTH}-\THEYEAR}
%\newdateformat{specialdate}{\twodigit{\THEDAY}-\THEYEAR}
\date{\specialdate\today}

\newcommand{\HRule}{\rule{\linewidth}{0.25mm}}


% CONSTANTES NECESARIAS PARA EL DOCUMENTO ---> MODIFIQUEN A SU CRITERIO
%\newcommand{\matricula}           {1430044}  %SU MATRICULA
\newcommand{\elola}           {el}  %Hombres cambien LA por EL
\newcommand{\OA}           {o}  %Hombres cambien A por O
\newcommand{\ncuatrimestre}{septiembre-diciembre 2017}
\newcommand{\nproyecto}           {Nombre del proyecto}
% NOTA: Dos diagonales juntas (\\) indican un saldo de linea. En este caso particular hay 2 (el titulo se ajusta a tres lineas, porque es muy largo. Hacer las adecuaciones pertinentes
\newcommand{\nproyectoheader}     {Nombre del proyecto}
\newcommand{\nalumno}             {Jesús Antonio Luna Alvarez}
\newcommand{\ncarrera}            {Ingeniería Mecatrónica}
\newcommand{\nasesorinstitucional}{Dr. Marco Aurelio Nuño Maganda}
\newcommand{\nasesorempresaria}   {Dr. Said Polanco Martagón}
\newcommand{\organismoreceptor}   {Aptiv de México, S. A. de C. V.}
\newcommand{\fecha}               {Mayo de 2019}
\newcommand{\separacionCorta}{0.0cm}
\newcommand{\separacionLarga}{0.5cm}

\usepackage[overload]{textcase}
\newcommand{\iemph}[1]{\MakeTextUppercase{#1}}

\pagestyle{fancy}
\headheight 35pt
\fancyhead{} % Clear all header fields
%\fancyhead[L]{\includegraphics[height=1.5cm]{CGUTP2_2019.jpg}}%
\fancyhead[C]{\nproyectoheader}%
%\fancyhead[R]{\includegraphics[height=1.5cm]{LogoUPV_2019.png}}%
\fancyfoot[R]{\thepage} % Clear all footer fields 
\fancyfoot[C]{}
\fancyfoot[L]{}

\DefineBibliographyStrings{english}{%
  references = {Referencias},% replace "references" with "bibliography"  for `book`/`report`
}

\addto\captionsenglish{%
  \renewcommand{\figurename}{Figura}%
  \renewcommand{\tablename}{Cuadro}%
} 

\usepackage{wallpaper}
 
 
 
\begin{document}

%-----------------------------------------------------------------------------------------------------------------
% PAGINA 1 - PORTADA
\setcounter{page}{1}
\pagenumbering{roman}
\thispagestyle{empty}

\begin{center}

\begin{tabular}{cp{8cm}c}
\includegraphics[height=2.25cm]{imgs/CGUTP2.png} & 
& \includegraphics[height=2.25cm]{imgs/LogoUPV_2019.png}   \\
\end{tabular}

\Large{ \textbf{UNIVERSIDAD POLITÉCNICA DE VICTORIA} }
\vspace{0.5cm}
\hrule
\vspace{0.1cm} 
\hrule
\vspace{0.5cm}


%\HRule \\[\separacionCorta]
\textbf{\nproyecto} \\[\separacionLarga]
%\Large \textbf{TESINA}
%\HRule \\[\separacionLarga]
\vspace{0.5cm}
REPORTE DE ESTANCIA II \\  % <-------- Cambiar el número de estancia
\vspace{1.0cm}
Presentado por: \\[\separacionCorta]
\textbf{\nalumno}\\
\vspace{0.5cm}
Alumno de la carrera de: \\[\separacionCorta]
\textbf{\ncarrera} \\[\separacionLarga]
\vspace{1.0cm}
Asesor institucional: \\[\separacionCorta]
\textbf{\nasesorinstitucional} \\[\separacionCorta]
\vspace{0.5cm}
Asesor empresarial: \\[\separacionCorta]
\textbf{\nasesorempresaria} \\[\separacionCorta]
\vspace{0.5cm}
Organismo receptor: \\[\separacionCorta]
\textbf{\organismoreceptor} \\[\separacionLarga]
\vspace{0.5cm}

\end{center}
\begin{flushright}
Ciudad Victoria, Tamaulipas, \fecha
\end{flushright}

\HRule 
%-----------------------------------------------------------------------------------------------------------------

%-----------------------------------------------------------------------------------------------------------------
% PAGINA II - RESUMEN EN ESPAÑOL Y EN INGLÉS

\clearpage
\section*{\centering Resumen}
\addcontentsline{toc}{section}{Resumen}
\input{secciones/01Resumen.tex}

\section*{\centering Abstract}
\addcontentsline{toc}{section}{Abstract}
\input{secciones/02Abstract.tex}\\
\\
\textbf{Palabras clave:} Red Neuronal Artificial, Clasificador, Segmentación, Puntos Clave, Regiones de Interés.

%-----------------------------------------------------------------------------------------------------------------
% PAGINA III - INDICE

\clearpage
\addcontentsline{toc}{section}{Índice}
\renewcommand\contentsname{Índice}
\tableofcontents

%-----------------------------------------------------------------------------------------------------------------
% CAPITULOS

\clearpage
\pagenumbering{arabic}
\setcounter{page}{1}
\input{secciones/03Introduccion.tex}

\clearpage
\input{secciones/04MarcoTeorico.tex}

\clearpage
\input{secciones/05JustificacionObjetivos.tex}

\clearpage
\input{secciones/06DesarrolloDelProyecto.tex}

\clearpage
\input{secciones/07Resultados.tex}

\clearpage
\input{secciones/08Conclusiones.tex}

%-----------------------------------------------------------------------------------------------------------------
% REFERENCIAS


\clearpage
%Let's cite! The Einstein's journal paper \cite{dirac} and the Dirac's 
%book \cite{einstein} are physics related items. 

%\Urlmuskip=0mu plus 1mu\relax
\addcontentsline{toc}{section}{Referencias} 
\printbibliography

\end{document}
